%!TEX TS-program = xelatex
%!TEX encoding = UTF-8 Unicode

\documentclass[$fontsize$, a4paper]{article}

% LAYOUT
%--------------------------------
\usepackage{geometry}
\geometry{$geometry$}

% No page numbers
\pagenumbering{gobble}

% Left align
\usepackage[document]{ragged2e}

\usepackage{setspace}

$if(letterhead)$
  % To include the letterhead
  \usepackage{wallpaper}
  \ULCornerWallPaper{1}{letterhead.pdf}
$endif$

% TYPOGRAPHY
%--------------------------------
\usepackage{fontspec}
\usepackage{xunicode}
\usepackage{xltxtra}

% converts LaTeX specials (quotes, dashes etc.) to Unicode
\defaultfontfeatures{Mapping=tex-text}

% Set paragraph break
\setlength{\parskip}{1em}

$if(seriffont)$
\setmainfont[SmallCapsFeatures={LetterSpace=5,Letters=SmallCaps}]{$seriffont$}
$endif$
$if(sansfont)$
    \setsansfont{$sansfont$}
$endif$

% Command required by how Pandoc handles the list conversion
\providecommand{\tightlist}{%
  \setlength{\itemsep}{0pt}\setlength{\parskip}{0pt}}

% TABLE CUSTOMIZATION
%--------------------------------
\usepackage{tabularx}
\usepackage{spreadtab}
\usepackage[compact]{titlesec} % For customizing title sections
\titlespacing*{\section}{0pt}{3pt}{-7pt} % Remove margin bottom from the title
\usepackage{arydshln} % For the dotted line on the table
\renewcommand{\arraystretch}{1.5} % Apply vertical padding to table cells
\usepackage{hhline} % For single-cell borders
\usepackage{enumitem} % For customizing lists
\setlist{nolistsep} % No whitespace around list items
\setlist[itemize]{leftmargin=0.5cm} % Reduce list left indent
\setlength{\tabcolsep}{9pt} % Larger gutter between columns


% LANGUAGE
%--------------------------------
$if(lang)$
\ifnum 0\ifxetex 1\fi\ifluatex 1\fi=0 % if pdftex
  \usepackage[shorthands=off,$for(babel-otherlangs)$$babel-otherlangs$,$endfor$main=$babel-lang$]{babel}
$if(babel-newcommands)$
  $babel-newcommands$
$endif$
\else
  % load polyglossia as late as possible as it *could* call bidi if RTL lang (e.g. Hebrew or Arabic)
  \usepackage{polyglossia}
  \setmainlanguage[$polyglossia-lang.options$]{$polyglossia-lang.name$}
$for(polyglossia-otherlangs)$
  \setotherlanguage[$polyglossia-otherlangs.options$]{$polyglossia-otherlangs.name$}
$endfor$
\fi
$endif$

% PDF SETUP
%--------------------------------
\usepackage[xetex, bookmarks, colorlinks, breaklinks]{hyperref}
\hypersetup
{
  pdfauthor=$author$,
  pdfsubject=Facture \#$invoice-nr$,
  pdftitle=Facture \#$invoice-nr$,
  linkcolor=blue,
  citecolor=blue,
  filecolor=black,
  urlcolor=blue
}

% DOCUMENT
%--------------------------------
\begin{document}
\small
\textsc{\textbf{$author$}}
$for(from)$
\textbullet{} \textsc{$from$}
$endfor$

\vspace{1em}

\fbox{
  \begin{minipage}{0.5\textwidth}
    \vspace{0.5em}
    \small \sffamily \onehalfspacing
    $for(to)$
    $to$\\
    $endfor$
    \vspace{0.5em}
  \end{minipage}
}

\vspace{3em}

\begin{flushright}
  \small
  $city$, le $if(date)$$date$$else$\today$endif$
\end{flushright}

\vspace{1em}


\section*{\textsc{Facture} \textsc{\#$invoice-nr$}}
\footnotesize
\newcounter{pos}
\setcounter{pos}{0}
\STautoround*{2} % Get spreadtab to always display the decimal part
$if(commasep)$\STsetdecimalsep{,}$endif$ % Use comma as decimal separator

\sffamily

\begin{spreadtab}{{tabularx}{\textwidth}{lXrrr}}
  \hdashline[1pt/1pt]
  @ \noalign{\vskip 2mm} \textbf{\#} & @ \textbf{Description} & @ \textbf{$if(unit.name)$$unit.name$$else$Quantité$endif$} & @ \textbf{Prix} \\ \hline
$for(service)$
          @ \noalign{\vskip 2mm} \refstepcounter{pos} \thepos
        & @ $service.description$
        $if(service.details)$\newline \begin{itemize}
          $for(service.details)$\scriptsize \item $service.details$
          $endfor$ \end{itemize}
        $endif$& :={$service.quantity$}$unit.suffix$
        & :={[-1,0] * $rate$}€ \\
$endfor$
  \noalign{\vskip 2mm} \hline & @ \multicolumn{1}{r}{\textbf{Total:}}
  & \textbf{:={sum(c1:[0,-1])}$unit.suffix$}
  & \textbf{:={sum(d1:[0,-1])}€} \\ \hhline{~~--}
\end{spreadtab}

\vspace{10mm}

$body$

\vspace{5mm}

$if(payment)$
{
  \itshape
  $for(payment.conditions)$
  $payment.conditions$ \\
  $endfor$
}
\paragraph{Coordonnées bancaires} ~\\[5pt]

\sffamily
\begin{tabular}{rl}
  \textbf{Titulaire} & $payment.holder$ \\
  \textbf{RIB}       & $payment.rib$ \\
  \textbf{IBAN}      & $payment.iban$ \\
  \textbf{BIC}       & $payment.bic$ \\
\end{tabular}
$endif$


\vfill
\rule{\textwidth}{1pt}

\begin{minipage}{\textwidth}
  \footnotesize
  \itshape
  \sffamily
  \onehalfspacing
  \centering
  $for(footer)$
  $footer$ \\
  $endfor$
\end{minipage}

\end{document}
